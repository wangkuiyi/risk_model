%!TEX encoding = UTF-8 Unicode
\documentclass[12pt,a4paper]{article}
\usepackage{amsmath}
\usepackage{xltxtra,fontspec,xunicode}
\usepackage[slantfont,boldfont]{xeCJK} % 允许斜体和粗体

\setCJKmainfont{Kai}   % 设置缺省中文字体
\setCJKmonofont{Hei}   % 设置等宽字体
\setmainfont{Optima}   % 英文衬线字体
\setmonofont{Monaco}   % 英文等宽字体
\setsansfont{Trebuchet MS} % 英文无衬线字体

\newcommand{\V}{\boldsymbol}

\begin{document}
\title{一个能核实互联网用户身份并利用互联网用户信息的Credit Risk Model}
\author{王益}
\maketitle

\section{问题}

金融机构的贷款历史数据可以用来训练一个credit risk model,比如logistic regression model,用来预估一个贷款者的信誉(本文中简化为还款率)。但是金融机构了解的贷款者属性不够丰富;我们希望从互联网上找到贷款者的其他信息,让credit risk model的特征更加丰富,从而预估更加精准。但是一个难处是,我们不确定一个在金融机构注册的贷款者是不是某个互联网用户。这里我们提出一个具备聚类特点的credit risk model,它既学习预估一个人的还款率,同时匹配贷款者和互联网用户的身份。

\section{数据}

考虑金融机构有$N$条贷款记录,每一条记录$\langle r_i, y_i\rangle$中,$r_i$是一个布尔变量,如果是1则表示还款了,否则用0表示没有还款;$y_i$是一个贷款者。通常包括姓名、年龄、身份证号等信息。这是训练数据。另外,存在从互联网上爬下来了$M$个互联网用户的信息,记为${u_j}$, $1\leq j \leq M$。每个$u_j$里的信息通常比$y_i$里的丰富,比如包括最近在微博上说了什么,最近和哪些人有过交流等。

\section{模型}

我们希望核实互联网用户身份。具体的说,想知道每个$y_i$很可能对应哪个$u_j$。然后希望利用$u_j$里丰富的信息,帮助我们训练一个精确的creidit risk model。为此,我们为每一个$y_i$增加一个hidden variable $z_i\in[1,M]$,用来标记$y_i$对应哪个$u_j$。这样,预估一个新的$y$对应的$r$的问题就变成了这样:
%
\begin{equation}
\label{eq:model}
P(r|y) = \sum_{1\leq z\leq M} P(r|u_z) P(z|y)
\end{equation}
%
我们得到了一个隐含变量模型,可以用EM算法来训练之。

我们假设$P(r|u_z)$用一个logistic regression model来描述,但是实际上用什么模型都可以,下面所述算法都能支持。而$P(z|y)$表示成一个$M$维向量$\V\gamma_y$。

\section{训练}

有隐含变量的模型,通常用EM算法来学习。EM算法是一个有收敛性保证的meta-algorithm——只要我们循环执行一个E-step和一个M-step,就能得到一个单调收敛的模型。其中E-step是估计hidden variable的概率分布,而M-step可以利用E-step的结果,通过最大化模型的log-likelihood来更新模型参数。为此我们在下文中推导hidden variable的分布计算公式和模型的log-likelihood的最大化方法。

\subsection{初始化}

如果我们完全不知道$y_i$应该如何和$u_j$对应,那么我们只能假设$z_i$的分布是$[1,M]$区间上的均匀分布。此时初始化就是把每个$\V\gamma_{y_i}$的每个元素都设置为$1/M$。

\subsection{E-step}

更新每个$\gamma_{y_i}$,其中$\gamma_{y_i,j}=P(z_i=j \mid r_i, y_i)$,而

\begin{equation}
\label{eq:estep}
P(z=j \mid r,y) = \frac{P(z,r \mid y)}{P(r\mid y)} = \frac{P(r|u_j) P(z=j|y)}{\sum_{1\leq z\leq M} P(r|u_z) P(z|y)}
\end{equation}

\subsection{M-step}

如果$P(r|u)$是用logistic regression model描述,那么会有一组参数$\V\beta$。严格的M-step要最大化log-likelihood:
\begin{equation}
\label{eq:logll}
\V\beta^* = \arg\max_{\V\beta} L(\V\beta) = \arg\max_{\V\beta} \sum_{1\leq i\leq N} \log P(r_i|y_i; \V\beta) 
\end{equation}
%
其中
\begin{equation}
\log P(r_i|y_i; \V\beta) 
= \sum_{1\leq i\leq N} \log \sum_{1\leq z\leq M} P(r_i|u_z) \gamma_{y_i,z}
\end{equation}

这个两个$\sum$之间夹着一个$\log$的形式很不容易对$\V\beta$求导。但是我们可以做一个简化,把第二个$\sum$去掉:

\begin{equation}
\label{eq:simplify}
\log P(r_i|y_i; \V\beta) 
\approx \sum_{1\leq i\leq N} \log P(r_i|u_x) 
\end{equation}
%
其中$x=\arg\max_j \gamma_{y_i,j}$。

这个近似和K-mean算法对EM clustering算法的近似很相近。它的好处是,把M-step变成了调用标准logistic regression model训练算法,甚至可以给logistic regression model加上L-1/L-2 regularization。

\section{改进}

从算法原理上,大家都知道EM算法会很容易陷入局部最优。对这个问题尤其如此。假设:
%
\begin{enumerate}
\item 有两个贷款者:Bob和Alice,其中Bob总不还贷,Alice总还贷;
\item 有两个互联网用户:一男一女;
\item 用户特征只有两个:gender=male和gender=female;
\item 随机初始化时,恰好Bob和Alice都被认为非常像那个男的互联网用户。
\end{enumerate}

那么M-step会学得一个logistic regression model,它认为gender=male这个特征的weight接近0,因为从Alice和Bob的还款记录来看——有的还款有的不还。另外gender=male这个特征的weight也是0,因为这个特征根本没有出现在logistic regression model的训练数据里。这样一来,这个模型总是认为还款率是50\%。等到E-step时,公式(\ref{eq:estep})里的$P(r|u_j)$总是0.5,那么$P(z=j\mid r,y)$主要受到$P(z=j|y)$的影响,而后者就是随机初始化的结果。换句话说,M-step对logistic regression model的更新没法帮助我们修正随机初始化中的错误。

一个直接的解法是公式(\ref{eq:simplify})里的$x$的取法从$x=\arg\max_j\gamma_{y_i,j}$改成$x\sim Discrete(\V\gamma_{y_i})$。也就是从“找概率最大的互联网用户”变成“按照分布$\V\gamma_{y_i}$随机选择一位互联网用户“。这样就把$x$的分布的估计改成了Gibbs sampling。那么上述EM算法也就成了一个stochastic EM算法。

\end{document}
